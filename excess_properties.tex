\documentclass[review]{elsarticle}

\usepackage{lineno,hyperref}
\usepackage{pdflscape}
\modulolinenumbers[5]

\journal{Geochemica and Cosmochemica Acta}

%%%%%%%%%%%%%%%%%%%%%%%
%% Elsevier bibliography styles
%%%%%%%%%%%%%%%%%%%%%%%
%% To change the style, put a % in front of the second line of the current style and
%% remove the % from the second line of the style you would like to use.
%%%%%%%%%%%%%%%%%%%%%%%

%% Numbered
%\bibliographystyle{model1-num-names}

%% Numbered without titles
%\bibliographystyle{model1a-num-names}

%% Harvard
\bibliographystyle{model2-names.bst}\biboptions{authoryear}

%% Vancouver numbered
%\usepackage{numcompress}\bibliographystyle{model3-num-names}

%% Vancouver name/year
%\usepackage{numcompress}\bibliographystyle{model4-names}\biboptions{authoryear}

%% APA style
%\bibliographystyle{model5-names}\biboptions{authoryear}

%% AMA style
%\usepackage{numcompress}\bibliographystyle{model6-num-names}

%% `Elsevier LaTeX' style
%\bibliographystyle{elsarticle-num}
%%%%%%%%%%%%%%%%%%%%%%%

\begin{document}

\begin{frontmatter}

\title{Excess properties of solid solutions: Modelling phase relations and seismic velocities in the mantle}
%\tnotetext[mytitlenote]{Fully documented templates are available in the elsarticle package on \href{http://www.ctan.org/tex-archive/macros/latex/contrib/elsarticle}{CTAN}.}

%% Group authors per affiliation:
\author{R. Myhill, C. Beyer}
\address{Bayerisches Geoinstitut, Universit\"{a}t Bayreuth, Universit\"{a}tsstrasse 30, 95447 Bayreuth, Germany}
\cortext[mycorrespondingauthor]{Corresponding author: R. Myhill}
\ead{myhill.bob@gmail.com}

\begin{abstract}

\end{abstract}

\begin{keyword}
high pressure \sep excess properties
\end{keyword}

\end{frontmatter}

\linenumbers

\section{Introduction}

\section{Motivation}
\subsection{Bulk moduli}
The regular solution model implicitly defines many of the thermoelastic properties of solid solutions. For example, it defines the bulk modulus as the Reuss average of the endmembers:

\begin{eqnarray}
K_T &=& -V\left( \frac{\partial P}{\partial V}\right) \\
 &=& \left( \sum_i n_i V_{i} + V_{\textrm{excess}}\right) \left( \frac{1}{\sum_i\left(n_i \frac{V_{i}}{K_{Ti}} \right)} + \frac{\partial P}{\partial V_{\textrm{excess}}} \right) 
\end{eqnarray}

\noindent $V_{\textrm{excess}}$ is often approximated as zero, or as a scalar term with no pressure dependence. There are potential problems with this approximation:
\begin{itemize}
\item Excess enthalpy terms may change sign and/or become very large as pressure increases. For example, pyrope-grossular in Green et al., 2012 has an enthalpy interaction of 31 kJ/mol, and an excess volume of 0.164 kJ/kbar/mol, leading to a doubling of enthalpy over its stability field. 
\item The model predicts an excess bulk modulus with the same sign as the excess volume, conflicting with the Anderson-Anderson (1970) rule of thumb that $V\, K_T = c$ 
\item Higher order excess volume terms to fit excess bulk moduli cause large deviations from a given equation of state at high pressure, as polynomial expansions of equations of state converge very slowly.
\end{itemize}

\noindent In many circumstances, these shortcomings are unimportant. However, there is a fundamental need to adjust the solid solution model where
\begin{itemize}
\item the stability range of the solid solution extends over pressure regimes large enough to contribute significantly to changes in excess enthalpy inconsistent with data, or
\item mineral phases exhibit significant excess bulk moduli. The latter is especially important for those studies where velocities are obtained from the models.
\end{itemize}

\subsection{Shear moduli}
Thermodynamic variables do not include any input/constraints on shear moduli, which appear to deviate significantly from Voigt and Reuss averages.

\section{Strategies}
Taylor expansion of EoS converges extremely slowly, and leads to rapid divergence at pressures greater than the last crossing. This is part of the reason that \cite{SLB2011} set excess volumes as zero. Thus, simply adding a bulk modulus term is inappropriate.

Best fit elastic properties for an ideal intermediate can be found, from which deviations can be applied.  


\section{Excess properties}
Most existing solid solution models define an excess enthalpy, entropy and volume, which leads to the following expressions for thermodynamic potentials:
\begin{eqnarray}
\mathcal{H}_{\textrm{ss}} &= \sum_in_i\mathcal{H}_i + \mathcal{H}_{\textrm{excess}} + PV_{\textrm{excess}}\\
\mathcal{S}_{\textrm{ss}} &= \sum_in_i\mathcal{S}_i + \mathcal{S}_{\textrm{conf}} + \mathcal{S}_{\textrm{excess}} \\
\mathcal{G}_{\textrm{ss}} &= \mathcal{H}_{\textrm{ss}} - T\mathcal{S}_{\textrm{ss}}\\
V_{\textrm{ss}} &= \sum_in_iV_i + V_{\textrm{excess}}
\end{eqnarray}
The derivatives of volume with respect to pressure and temperature can then be calculated
\begin{eqnarray}
\alpha_{P,\textrm{ss}} &= \frac{1}{V}\left(\frac{\partial V}{\partial T}\right)_P = \left( \frac{1}{V_{\textrm{ss}}}\right)\left( \sum_i\left(n_i\,\alpha_i\,V_i \right) \right) \\
K_{T,\textrm{ss}} &= -V\left( \frac{\partial P}{\partial V} \right)_T = V_{\textrm{ss}} \left( \left(\sum_i \frac{n_i V_{i}}{K_{Ti}} \right)^{-1} - \frac{\partial P}{\partial V_{\textrm{excess}}} \right) \\
K'_{T,\textrm{ss}} &= -\frac{\partial}{\partial P} \left (V\left( \frac{\partial P}{\partial V} \right)_T \right) \sim V_{\textrm{ss}} \left(\sum_i \frac{n_i V_{i}}{K'_{Ti} + 1.} \right)^{-1}
\end{eqnarray}

Making the approximation that the excess entropy has no temperature dependence
\begin{eqnarray}
C_{P,\textrm{ss}} &= \sum_in_iC_{Pi}\\
C_{V, \textrm{ss}} &= C_{P,\textrm{ss}} - V_{\textrm{ss}}\,T\,\alpha_{\textrm{ss}}^{2}\,K_{T,\textrm{ss}} \\
K_{S,\textrm{ss}} &= K_{T,\textrm{ss}} \,\frac{C_{P,\textrm{ss}}}{C_{V,\textrm{ss}}}\\
\gamma_{\textrm{ss}} &= \frac{\alpha_{\textrm{ss}}\,K_{T,\textrm{ss}}\,V_{\textrm{ss}}}{C_{V, \textrm{ss}}}
\end{eqnarray}

\section{Conclusions}



\clearpage
\section*{References}

\bibliography{references_xs}

\end{document}
